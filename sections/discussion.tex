% !TeX root = ../paper.tex
% !TeX encoding = UTF-8
% !TeX spellcheck = en_US

\section{Discussion}\label{sec:discussion}
  As we have described in \cref{sec:self-healing-kubernetes} \gls{kubernetes} provides all features required to setup a self-healing systems.
  Nevertheless, it requires the managed application to make use of the features and the user (developer or administrator) to configure the application deployment correctly.
  This includes packing the application parts in containers, writing an application that supports replication and distribution, using \glspl{service}, \glspl{pod controller}, \glspl{priority class}, and \glspl{pdb}, and setting up external services for persistent storage of stateful services.
  Compared to implementing new recovery methods or specifying imperative recovery policies, \gls{kubernetes} is configured via the declarative definition of the desired system state.
  This provides a high abstraction and eases the usage of \gls{kubernetes} for equipping a system with self-healing capabilities in a cloud environment.
  The recovery and healing logic is provided by \gls{kubernetes} and thoroughly tested in big production environments.
  \Gls{kubernetes} also provides a rich API to retrieve the current system state and update the desired state.
  This can be used to extend \gls{kubernetes} with further self-healing logic.
  Using \gls{kubernetes} for self-healing systems also comes with some limitations:
  \begin{itemize}
    \item \gls{kubernetes} has only an external view on the managed system.
      This is no problem at all, because failure recovery in the cloud domain is reduced to detecting nodes unreachable and re-deploying the software that was running on the node on another one (see \cref{sec:self-healing:cloud}).
    \item \gls{kubernetes} does not automatically repair failing infrastructure, such as nodes, external load balancers, or storage volumes.
      It relies on the availability and fault tolerance of the underlying infrastructure for \texttt{PersistentVolumes}.
      % --> automatic node repairs on GCE: \url{https://cloud.google.com/kubernetes-engine/docs/how-to/node-auto-repair}
    \item As \gls{kubernetes} is external to the application, it must itself be resilient and fault tolerant.
      Per default the \gls{kubernetes}' master components are only deployed on one node.
      This does not provide any fault-tolerance and therefore one must deploy a high-availability \gls{kubernetes} setup with multiple master nodes, potentially across availability zones.
      This setup is considerably complex and not yet automated.
  \end{itemize}
