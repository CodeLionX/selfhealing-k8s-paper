% !TeX root = ../paper.tex
% !TeX encoding = UTF-8
% !TeX spellcheck = en_US

\section{Discussion}\label{sec:discussion}
  As described in \cref{sec:self-healing-kubernetes}, \gls{kubernetes} provides all features required to setup a self-healing system.
  Nevertheless, it requires the managed application to make use of the provided features and the user to configure the deployment correctly.
  This includes packing the application parts in containers, writing an application that supports replication and distribution, using \glspl{service}, \glspl{pod controller}, \glspl{priority class}, and \glspl{pdb}, and setting up external services for persistent storage of stateful services.
  Compared to implementing new recovery methods or specifying imperative recovery policies, \gls{kubernetes} is configured via the declarative definition of the desired system state.
  This provides a high abstraction and eases the usage of \gls{kubernetes} for equipping a system with self-healing capabilities in a cloud environment.
  The recovery and healing logic is provided by \gls{kubernetes} and thoroughly tested in big production systems.
  \Gls{kubernetes} also provides a rich API to retrieve the current system state and update the desired state.
  This can be used to extend \gls{kubernetes} with further self-healing logic.
  However, using \gls{kubernetes} for self-healing also comes with some limitations:
  \begin{itemize}
    \item \Gls{kubernetes} is an external self-healing manager and \citeauthor{ToffettiMicroservices} argue that \enquote{this approach has intrinsic limits} and requires additional management effort and human intervention and lead to vendor lock-in~\cite{ToffettiMicroservices}.
    \item \Gls{kubernetes} does not automatically repair failing infrastructure, such as nodes, external load balancers, or \texttt{PersistentVolumes}.
      It relies on the availability and fault tolerance of the underlying infrastructure.
      % --> automatic node repairs on GCE: \url{https://cloud.google.com/kubernetes-engine/docs/how-to/node-auto-repair}
    \item As \gls{kubernetes} is external to the application, it must itself be resilient and fault tolerant.
      By default, the \gls{kubernetes}' master components are only deployed on one node.
      This does not provide any fault-tolerance.
      To achieve this, \glspl{kubernetes} must be deployed in a highly available setup with multiple master nodes, potentially across availability zones.
      This setup is considerably complex and not yet automated.
  \end{itemize}
