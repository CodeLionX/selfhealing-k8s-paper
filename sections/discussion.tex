% !TeX root = ../paper.tex
% !TeX encoding = UTF-8
% !TeX spellcheck = en_US

\section{Discussion}\label{sec:discussion}
  \begin{enumerate}
    \item requires containerized microservice application
    \item code must support scaling and dynamic communication
    \item provider of \texttt{PersistentVolumes} must ensure their availability and fault-tolerance
    \item to deal with a node failure, remaining nodes must have enough spare capacity to host the failed pods
    \item with replication factor 1, there are down times during re-creation of the pod on another node
    \item limitations
    \begin{itemize}
      \item \textbf{external management logic has to be themselves resilient, fault-tolerant, and scalable}
      \item \gls{kubernetes} default only one master $\rightarrow$ HA setup across availability zones
      \item quite a lot of configuration work, not automation yet (WIP)
      \item only one master will be active (the other two will be passive), full state replication via etcd
      \item fail-over will be handled by load balancer component
      \item \textbf{only external view on the system}
      \item \textbf{\gls{kubernetes} does not automatically repair or restart failing nodes}
      \item --> automatic node repairs on GCE: \url{https://cloud.google.com/kubernetes-engine/docs/how-to/node-auto-repair}
      \item components external to \gls{kubernetes} are not included in self-healing logic (such as external storage or load balancers of cloud provider)
    \end{itemize}
    \item benefits
    \begin{itemize}
      \item healing from pod / container failures and node failures out-of-the-box
      \item declarative definition of system state
      \item rich API to retrieve current system state
    \end{itemize}
    \item interesting facts and insights
  \end{enumerate}