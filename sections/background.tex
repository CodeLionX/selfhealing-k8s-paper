% !TeX root = ../paper.tex
% !TeX encoding = UTF-8
% !TeX spellcheck = en_US

\section{Self-Healing}
  \begin{enumerate}
    \item sub control loop of MAPE-K loop (Detect -- Analyze -- Recover)~\cite{PsaierSurvey}
    \item different levels of self-healing (architecture-based, model-based, hierarchical, \etc)
    \item self-healing management logic external and internal to the managed application
      \begin{description}
        \item[external to application]\hfill\\
          \begin{itemize}
            \item self-healing and management logic is run in isolation from the application code
            \item Examples: using services from the infrastructure provider, using third party services, or building an ad-hoc solution (\eg using \gls{kubernetes})~\cite{ToffettiMicroservices}
            \item current state of the art for monitoring, health management, and scaling logic
            \item could lead to vendor lock-in
            \item external management logic has to be themselves resilient, fault-tolerant, and scalable
          \end{itemize}
        \item[within application]\hfill\\
        \begin{itemize}
          \item approach by~\citeauthor{ToffettiMicroservices} for microservices; leverages standard methods from distributed systems (such as consensus algorithms) to assign self-management functionality to nodes of the application; hierarchical approach~\cite{ToffettiMicroservices}
        \end{itemize}
      \end{description}
  \end{enumerate}

\section[Kubernetes]{\gls{kubernetes}}
  \begin{enumerate}
    \item what is \gls{kubernetes}? --> \url{https://kubernetes.io/docs/concepts/overview/what-is-kubernetes/}
    \item architecture and how it works
      \begin{itemize}
        \item master-slave architecture
        \item master runs kube controller manager, API server, etcd, kube scheduler, cloud controller manager
        \item slave (nodes) run kubelet (pod management and health monitoring) and kube proxy (cluster networking), and container runtime (e.g. Docker)
        \item only slaves run application code
      \end{itemize}
    \item \gls{kubernetes} objects\footnote{\url{https://kubernetes.io/docs/concepts/overview/working-with-objects/kubernetes-objects/}} and labels\footnote{\url{https://kubernetes.io/docs/concepts/overview/working-with-objects/labels/}}
    \item pods and containers\footnote{\url{https://kubernetes.io/docs/concepts/workloads/pods/pod-overview/}}
  \end{enumerate}
