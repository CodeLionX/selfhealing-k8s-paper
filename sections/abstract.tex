% !TeX root = ../paper.tex
% !TeX encoding = UTF-8
% !TeX spellcheck = en_US

% abstract
\begin{abstract}
  One essential part of a self-adaptive system is its self-healing capabilities.
  Self-healing systems monitor the running application and try to keep the system in a healthy state to increase availability and adapt to unexpected changes.
  Therefore, they have to be fault tolerant, mask temporary failures, maintain essential services, and recover from the failures in a finite amount of time to reach the healthy system state again.
  In cloud environments, self-healing techniques are already used in form of tools that try to achieve continuous availability for cloud services.\\
  In this paper, we take a look at the self-healing capabilities of \gls{kubernetes} for microservice architectures in the cloud and compare it to the approaches in self-healing literature.
  We find that \gls{kubernetes}' approach is a form of architecture-based self-healing and that \gls{kubernetes} implements all important aspects of self-healing systems.
  However, \gls{kubernetes} depends on the underlying infrastructure to provide fault-tolerant persistent volumes, must run in an highly available setup to be itself resilient in case of \gls{kubernetes} master component failures.
\end{abstract}