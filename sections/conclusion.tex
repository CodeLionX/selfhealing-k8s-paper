% !TeX root = ../paper.tex
% !TeX encoding = UTF-8
% !TeX spellcheck = en_US

\section{Conclusion}\label{sec:conclusion}
  In this work, we showed how \gls{kubernetes} can be used to equip a microservice application with self-healing capabilities.
  \Gls{kubernetes} monitors the cluster state and takes corrective actions to let the cluster converge to the user-defined, desired state.
  \Gls{kubernetes} implements fault-tolerance through the replication and isolation of application parts encapsulated in pods and load balancing the network accesses to the pods using \glspl{service}.
  On pod failures, other replicas can simply take over the failed node's work.
  After the failure, the cluster converges back to the desired state.
  \Glspl{pod controller} detect pod or node failures, diagnose them, and recover the failed pods by terminating the failed pod and scheduling a new identical one.
  This reflects the self-stabilizing aspect of self-healing systems.
  \Glspl{priority class} and \glspl{pdb} help the \gls{kubernetes} scheduler on resource pressure and occurring failures to maintain the essential services, while recovering the non-essential services when enough resources are available again.
  This allows the system to survive extreme stress situations.
  \Gls{kubernetes} depends on the underlying infrastructure to provide fault-tolerant \texttt{PersistentVolumes} and must be run in an high-availability setup to be itself resilient in the case of \gls{kubernetes} master component failures.