% !TeX root = ../paper.tex
% !TeX encoding = UTF-8
% !TeX spellcheck = en_US

\section{Introduction}\label{sec:introduction}

  Cloud Computing has become the de-facto standard of deploying new scalable applications.
  Companies chose cloud over on-premise or self-hosted environments, because they can deploy their applications more flexible, with higher and dynamically scalable performance, and because prices are very competitive~\cite{ToffettiMicroservices}.
  However, present cloud environments have to deal with heterogeneous resources and an ever-increasing scale.
  With this growing complexity failures are more likely to occur and software engineers have to design applications with that in mind.
  This can be achieved via replication, containment, isolation, and monitoring paired with responsive actions to failures~\cite{reactivemanifesto}.

  One way to realize containment and isolation of software components in a scalable way are microservice architectures.
  In this approach, the software application is decomposed along business domain boundaries into small, lightweight, autonomous services.
  Each service runs as its own application decoupled from the other services and acts as a scaling unit.
  Microservice architectures embrace failure.
  If a service relies on another services, it is aware that the other service may not be available or the connection may be slow and the service can deal with the failures~\cite{microservices}.

  For those complex distributed software systems consisting of hundreds of microservices, deployment and management gets more complex as well.
  That's the reason automation tools, such as \gls{kubernetes}~\cite{kubernetes}, for the deployment, scaling, and management of distributed applications exist.

  The increasing complexity of modern software systems motivated the development of self-adaptive systems.
  Those systems introduce an autonomous behavior that takes decisions at runtime and manages the complex underlying software system.
  This allows the software systems to adapt to unpredictable system changes and changing environments.
  Self-adaptive systems combine four self properties, as defined by \citeauthor{Ganek}~\cite{Ganek}:

  \begin{description}
    \item[self-configuring] The systems adapt automatically to dynamically changing environments (\enquote{on-the-fly}).
    \item[self-healing] The systems discover, diagnose, and react to failures reducing disruptions and enabling continuous availability.
    \item[self-optimization] Systems efficiently maximize resource utilization.
    \item[self-protection] Systems anticipate, detect, identify, and protect themselves from attacks.
  \end{description}

  This list has been continuously extended and the extended properties are now referred to as self-* properties~\cite{PsaierSurvey}.

  Self-healing is an integral part of self-adaptive systems and the focus of this paper.
  It combines properties of
  (i) fault-tolerant systems, which handle transient failures and mask permanent ones to ensure system availability,
  (ii) self-stabilizing systems, which are non-fault masking and converge to the legal state in a finite amount of time, and 
  (iii) survivable systems, which maintain essential services and recover non-essential ones after intrusions have been dealt with~\cite{PsaierSurvey}.
  A widely-used definition for self-healing systems is from \citeauthor{Ghosh}~\cite{Ghosh}:

  \begin{quote}
    The key focus [...] is that a self-healing system should recover from the abnormal (or \enquote{unhealthy}) state and return to the normative (\enquote{healthy}) state, and function as it was prior to disruption.
  \end{quote}

  This definition is very broad, but one can argue that the key aspect of self-healing systems are recovery oriented functionalities that bring the system back to the healthy state, which neither sole fault-tolerant systems nor sole survivable systems encompass~\cite{PsaierSurvey}.


\begin{enumerate}
  \item self-healing of microservices in cloud environments
  \begin{itemize}
    \item \citeauthor{PsaierSurvey} compare self-healing in cloud environments to the techniques that achieve continuous availability of the application~\cite{PsaierSurvey}:
    \begin{quote}
      In cloud environments, self-healing can be considered as the techniques to achieve continuous availability, which involves detecting disruptions, diagnosing failures and recovering with a sound strategy.
    \end{quote}
    \item In a cloud environment and a VM or container deployment, all failures are reduced to a single one: service unavailable after a while
  \end{itemize}
\end{enumerate}