% !TeX root = ../paper.tex
% !TeX encoding = UTF-8
% !TeX spellcheck = en_US

\section{Introduction}\label{sec:introduction}

\begin{enumerate}
  \item cloud
  \begin{itemize}
    \item cloud computing de-facto standard in industry
    \item reasons: more flexibility, higher and dynamically available performance, and competitive prices~\cite{ToffettiMicroservices}
    \item more hardware means more hardware can fail --> plan for failure~\cite{microservices}
    \item need for resilient systems~\cite{reactivemanifesto}
    \item achieve it via replication, containment, isolation, and monitoring paired with responsive actions to failures
  \end{itemize}
  
  \item microservices
  \begin{itemize}
    \item way to allow scaling of applications combined with a way to realize containment and isolation on business boundaries
    \item decompose software application into small, lightweight, autonomous services ( = scaling units)
    \item embrace failure: services relying on other services should deal with them failing~\cite{microservices}
  \end{itemize}
  
  \item deployment and orchestration $\rightarrow$ \gls{kubernetes}
  
  \item self-adaptive systems
  \begin{itemize}
    \item self-* properties
    \item MAPE-K loop
  \end{itemize}
  
  \item self-healing
  \begin{itemize}
    \item currently widely-used definition for self-healing systems is from \citeauthor{Ghosh}~\cite{Ghosh}:
    \begin{quote}
      The key focus [...] is that a self-healing system should recover from the abnormal (or \enquote{unhealthy}) state and return to the normative (\enquote{healthy}) state, and function as it was prior to disruption.
    \end{quote}
    
    \item Neither fault-tolerant systems, nor survivable systems include recovery oriented functionalities that bring the system back to the healthy state, which is the key aspect of self-healing systems~\cite{Ghosh}.    
    \item Combination of~\cite{PsaierSurvey}
    \begin{itemize}
      \item Fault-tolerant (handle transient failures and mask permanent ones)
      \item self-stabilizing (non-fault masking; system converges to legal state in finite time and tries to remain in the same (closure))
      \item survivable (maintain essential service and recover non-essential after intrusions have been dealt with)
    \end{itemize}
  \end{itemize}
  
  \item self-healing of microservices in cloud environments
  \begin{itemize}
    \item \citeauthor{PsaierSurvey} compare self-healing in cloud environments to the techniques that achieve continuous availability of the application~\cite{PsaierSurvey}:
    \begin{quote}
      In cloud environments, self-healing can be considered as the techniques to achieve continuous availability, which involves detecting disruptions, diagnosing failures and recovering with a sound strategy.
    \end{quote}
    \item In a cloud environment and a VM or container deployment, all failures are reduced to a single one: service unavailable after a while
  \end{itemize}
\end{enumerate}