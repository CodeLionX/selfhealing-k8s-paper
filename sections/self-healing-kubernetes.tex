% !TeX root = ../paper.tex
% !TeX encoding = UTF-8
% !TeX spellcheck = en_US

\section{Using \gls{kubernetes} to implement a self-healing application}
  In this section we will go through the setup of a self-healing microservice application with \gls{kubernetes}.
  \gls{kubernetes}'s approach to self-healing is comparable to architecture-based self-healing~\cite{ToffettiMicroservices,DashofyArchitecture}.
  

  \begin{enumerate}
    \item How would a setup of a self-healing microservice architecture look like?
    \item comparable to architecture-based approach
      \begin{enumerate}
        \item \gls{kubernetes} object configuration corresponds to the desired runtime architecture of the managed application. (\cite{ToffettiMicroservices} call it \textit{instance graph})
        \item \gls{kubernetes} internally holds the current architecture of the running components (in \textit{etcd})
        \item Container failures are captured by the restart policy of their pods. When set to \textit{Always} or \textit{OnFailure}, failing containers are restarted with an exponential back-off delay\footnote{\url{https://kubernetes.io/docs/concepts/workloads/pods/pod-lifecycle/\#restart-policy}}.
        \item To deal with node failures, pods have to be managed by controllers (explained later)\footnote{\url{https://kubernetes.io/docs/concepts/workloads/pods/pod-lifecycle/\#pod-lifetime}}. They perform the Detect -- Analyze -- Recover loop by
          \begin{itemize}
            \item monitoring the health of their managed pods with heartbeats and user-defined liveness probes\footnote{\url{https://kubernetes.io/docs/tasks/configure-pod-container/configure-liveness-readiness-probes/}}
            \item comparing the desired and current state of their pods
            \item performing actions (create or delete pod) to transition into the desired state
          \end{itemize}
        \item \gls{kubernetes} sets the phase of all pods on a died or disconnected node to \textit{Failed}
      \end{enumerate}
    \item self-healing properties available in \gls{kubernetes} via controllers:\hfill\\
          \enquote{A Controller can create and manage multiple Pods for you, handling replication and rollout, and providing self-healing capabilities at cluster scope. -- \url{https://kubernetes.io/docs/concepts/workloads/pods/pod-overview/\#pods-and-controllers}}
          Pods are transparently placed on the available nodes by the controller.

          \begin{itemize}
            \item recovery of stateful applications:
              \begin{itemize}
                \item Deployment definition via \texttt{StatefulSet}: \url{https://kubernetes.io/docs/tutorials/stateful-application/basic-stateful-set/}
                \item  Uses \texttt{PersistentVolumes} (provided by the underlying cloud platform, e.g AWS, GCP, OpenStack) for storage
                \item Pods have a unique identity (name, network id, K8s configuration)
                \item Failed pods will be rescheduled on other nodes with their identity (re-using the assigned persistent volume and network id)
                \item A headless Service takes care of service discovery using SRV records and DNS (re-routing traffic to rescheduled pods on different nodes)
                \item therefore, relies on the availability and fault-tolerance of the used persistent volumes
              \end{itemize}

            \item recovery of stateless applications:
              \begin{itemize}
                \item Deployment definition via \texttt{Deployment} and the specification of replicas > 1 or with \texttt{ReplicaSet}
                \item Failing pods will be recreated to match the desired number of replicas (node placement is transparent)
              \end{itemize}

            \item daemons: applications per node
              \begin{itemize}
                \item Defined via \texttt{DaemonSets}: \url{https://kubernetes.io/docs/concepts/workloads/controllers/daemonset}
                \item Ensures (monitors, restarts) that a copy of an application is run on each node (also on added or removed nodes)
                \item no real recovery if a node fails. Relies on manual action to replace the failed node. Then the \texttt{DaemonSet} will take care of creating the daemon pod on the newly added node.
              \end{itemize}
          \end{itemize}
    \item regarding the survivability aspect of self-healing systems: \url{https://kubernetes.io/docs/concepts/configuration/pod-priority-preemption/}
      \begin{itemize}
        \item we can define priority classes and assign pods to those
        \item pod priority will affect scheduling order (higher priority pods first)
        \item under resource pressure, higher priority nodes that are created and scheduled will evict lower priority pods (with their graceful termination period after which they are killed)
        \item pod disruption budgets can be specified to limit the number of replicated pods that are simultaneously down from voluntary disruption (draining, and also preemption)\footnote{\url{https://kubernetes.io/docs/concepts/workloads/pods/disruptions/\#how-disruption-budgets-work}}
        \item pod disruption budgets are considered only on best effort basis during preemption
      \end{itemize}
  \end{enumerate}